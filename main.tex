%%% This is the Main tex document that performs the layout. 
%%% You will need to compile this using LuaLaTeX with the --shell-escape option

\documentclass[a4paper,twoside,11pt]{article}%% change size of font to fit in 8 pages

\newcommand{\lang}{english} %which language translation. %

%%%% some other preferences, not necessary to change.
\setlength\parindent{0cm} % Paragraph indentation
\usepackage[top=15mm, bottom=20mm, outer=15mm, inner=15mm]{geometry}% Margins
\usepackage[autocompile]{gregoriotex}% for layout of gregorian chant
\gresetlinecolor{gregoriocolor}%% red score lines in chants

\usepackage{fontspec}% necessary for gregorio
\usepackage{libertine}% the font %
\usepackage{xstring}% for calendar comparisons
\usepackage{paracol}% for parallel columns

\usepackage[latin,\lang]{babel} % the languages for hyphenation, etc.
\usepackage{vesperale} %% vespers style sheet has macros for typesetting
%\usepackage{calendar} %% for the selection of propria

%%%% some experimental preferences, not yet very effective:
%\setlength{\columnsep}{5mm}%% separation between columns
%\setlength{\columnseprule}{0.2pt}%% column rule

\begin{document}
  \begin{center}
    \huge{\textbf{Epiphania Domini}}\\
    \large{Ad I Vesperas}%%
  \end{center}

\section*{Initium}
\versiculus{or--deus_in_adjutorium--solesmes}  
\section*{Psalmi}
\psalmus{an--ante_luciferum_genitus--solesmes}{109-2D}
\psalmus{an--venit_lumen_tuum--solesmes}{110-1g2}
\psalmus{an--apertis_thesauris--solesmes}{111-1g2}
\psalmus{an--maria_et_flumina--solesmes}{112-4E}
\psalmus{an--stella_ista--solesmes}{116-7c2}
\section*{Capitulum}
\text{capitulum-5-jan}
\section*{Hymnus}
\hymnus{hy--crudelis_herodes_deum--solesmes}{reges-tharsis}{reges-arabum}
\section*{Canticum}
\psalmus{an--magi_videntes_stellam--solesmes}{Magnificat-8G}
\section*{Oratio}
\text{oratio}
\section*{Conclusio}
\versiculus{or--benedicamus_in_i_vesperis--solesmes}
\vskip .5cm
\text{fidelium}  

%%%  end of document%
\end{document}%assumptio/gabc/
%%% Emacs stuff %%%

%%% Local Variables:
%%% mode: latex
%%% TeX-master: t
%%% TeX-engine: luatex
%%% TeX-command-extra-options: "-shell-escape"
%%% End:
